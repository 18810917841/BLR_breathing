\documentclass{article}
\usepackage[left=1.5cm, right=1.5cm, top=3cm, bottom = 3cm]{geometry}
\usepackage{amsmath}
\usepackage{mathrsfs}
\usepackage{amsfonts}
\usepackage{amssymb}
\usepackage{graphicx}
\usepackage{float}
\usepackage{wrapfig}
\usepackage{latexsym}
\usepackage{hyperref}
\usepackage{feynmf}
\usepackage{exscale}
\usepackage{relsize}
\usepackage{bm}%bold math, for vector
\linespread{1.1}


\author{Yuyang Songsheng}
\title{The Variation of Radiation Pressure and Breathing of BLR}

\begin{document}
\maketitle
\section{Equation of motion}
Gravitation:
\begin{equation}
F_{grav} = -\frac{GMm}{r^2}
\end{equation}
Radiation pressure:
\begin{equation}
F_{rad} = \frac{GMm}{r^2} \cdot \frac{3l}{2\sigma_T N_{cl}}
\end{equation}
Here, $l$ is the ratio between actual luminosity $L$ and Eddington luminosity $L=\frac{4\pi GMm_pc}{\sigma_T}$.\\
If we suppose that the density of the gas in clouds varies with radius as
\begin{equation}
\rho_{cl} = \rho_{cl0} (\frac{r}{r_0})^{-s}
\end{equation}
then we know the column density varies as
\begin{equation}
N_{cl} = N_{cl0} (r/r_0)^{-\frac{2}{3}s}
\end{equation}
Drag force:
\begin{equation}
F_{drag} = -6\pi \mu R_{cl} v
\end{equation}
We suppose the viscosity $\mu$ varies as
\begin{equation}
\mu = \mu_0 (\frac{r}{r_0})^{\nu}
\end{equation}
The radius of clouds varies as 
\begin{equation}
R_{cl} = R_{cl0} (\frac{r}{r_0})^{\frac{s}{3}}
\end{equation}
To simplify our calculation, we must redefine our scales. we use the typical size of BLR $r_0$ as the scale of length and Kepler velocity $v_{0} = \sqrt{\frac{GM}{r_0}}$ as the scale of velocity. Then the acceleration of the cloud can be written as
\begin{equation}
a_{grav} = - \frac{1}{r^2}
\end{equation}
\begin{equation}
a_{rad} = \frac{\alpha f(t)}{r^{2-\frac{2}{3}s}}, \ \ \alpha = \frac{3l_0}{2\sigma_T N_{cl0}}
\end{equation}
(Here, the variation of light curve can be written as $l = l_0 *f(t)$)
\begin{equation}
a_{drag} = -\beta r^{\frac{s}{3}+\nu} v,\ \ \beta = \frac{6\pi \mu_0 R_{cl0} r_0}{m v_0}
\end{equation}
If we define the Reynolds number as
\begin{equation}
\mathrm{Re} = \frac{\rho u L}{\mu}
\end{equation}
Here, $u$ is the mean velocity of the cloud relative to the background gas and $L$ is a characteristic length, $\rho$ is the density of the background gas. After a few lines of algebra, we can get
\begin{equation}
\beta = \frac{\Lambda}{\mathrm{Re}_0},\ \ \Lambda = \frac{9}{2} \frac{r_0}{R_{cl0}} \frac{\rho_0}{\rho_{cl0}} \frac{u_0}{v_0} 
\end{equation}
As the drag force can not change the direction of angular momentum of the cloud, the motion of the cloud is constrained in a plane. We use polar coordinates to describe its motion.
\begin{equation}
\ddot{r} = \frac{L^2}{r^3} - \frac{1}{r^2} + \frac{\alpha f(t)}{r^{2-\frac{2}{3}s}} - \beta r^{\frac{s}{3}+\nu}\dot{r}
\end{equation}
\begin{equation}
\dot{L} = - \beta r^{\frac{s}{3}+\nu}L
\end{equation}



\section{The perturbation theory}
At first, we study the motion of clouds when the radiation pressure and drag force can be viewed as a perturbation,$i.e. \alpha \ll 1, \beta \ll 1$. The $0^{th}$ order approximation is just circular motion
\begin{equation}
r=1, \ \ L=1
\end{equation}
To the linear order,
\begin{equation}
\delta\ddot{r} = 
\end{equation}
\end{document}
